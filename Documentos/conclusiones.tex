\begin{comment}
\addbibresource{referencias/Referencias.bib}
\end{comment}

\section{Conclusiones}
\label{sc:conclusiones}

En primer lugar, gracias a los resultados obtenidos en el capítulo \ref{sc:resultados}, se puede afirmar que las funciones de atenuación tienen efecto en la solución, más aún, implican la obtención de un tipo de superficie de esfuerzos distinta. Es decir, la forma de la superficie varía dependiendo de la función de atenuación usada. Por tal razón, la función de atenuación debe ser un parámetro más para los modelos de elasticidad no local.

Al momento de seleccionar una función de atenuación se deben tener en cuenta los efectos que tenga dicha función en el método. Como se aprecia en el capítulo \ref{sc:resultados}, los resultados obtenidos con las funciones de atenuación biexponencial y modificada tienden a agruparse en zonas de deformaciones mayores, mientras que los resultados obtenidos con las otras dos funciones tienden a ser menores. Esta comparación es evidente en las zonas del dominio que se encuentran a una distancia menor a $Lr$ de los bordes.

La agrupación de los resultados que se aprecia en las figuras \ref{fig:perfilesY0019}, \ref{fig:perfilesY0259} y \ref{fig:perfilesbarraY05} se debe al origen de las funciones de atenuación. Por ejemplo, la función biexponencial y la función modificada están basadas en la función exponencial, más aún, la función propuesta es una modificación a la biexponencial, lo cual justifica que los resultados obtenidos sean similares. Este parecido es fuerte cuando el parámetro $l$ es pequeño. Estas dos funciones cuentan con una característica adicional que es relevante, dichas funciones no se definen a partir del parámetro $m_0$ (parámetro que define la magnitud de $Lr$). En estas funciones el parámetro $Lr$ se obtuvo a partir de inspeccionar los valores para los cuales la función será cercana a cero \parencite{Polizzotto2001}. En otras palabras, estas funciones independientes de $m_0$ obtienen $Lr$ a partir de sus valores.

Por otro lado, las funciones lineal y cuadrática son dependientes de $m_0$. En ellas, este parámetro adicional obliga a que la función sea cero a partir de una relación $\rho$ específica. Para poder realizar la comparación de resultados, se usó $m_0=5$ (lo cual lleva a un $Lr=6l$ para estas funciones de atenuación), sin embargo, los resultados obtenidos por estas funciones difieren con los obtenidos con las funciones independientes. Esta diferencia es relevante para valores de $l$ grandes.

Dado que el $Lr$ es un factor relevante que se obtiene al elegir una función de atenuación, se debe tener en cuenta que el enmallado producido debe ser afectado por este parámetro. Así como lo recomienda \textcite{Polizzotto2001} se recomienda un valor de $Lr=6l$ para la función biexponencial. Basados en la figura \ref{fig:modificada}, se recomienda un valor de $Lr=8l$ para la función modificada, de esta manera se asegura que la condición de la ecuación \ref{eq:condicion} se cumpla incluso para valores de $l$ grandes.

Adicionalmente, se debe tener en cuenta el costo computacional de la integración de dichas funciones. En el caso de las funciones basadas en la función exponencial, se necesitan una cantidad mayor de puntos de Gauss para obtener una precisión aceptable. Este problema escala el costo computacional de manera exponencial, ya que, el proceso de integración costoso se encuentra en los elementos no locales.
Por otro lado, dado que las funciones lineal y cuadrática son basadas en polinomios, se puede obtener el valor exacto de la integral usando pocos puntos de Gauss.

Como se mencionó en el capítulo \ref{sc:resultados}, en ciertos perfiles se evidencia la ceración de puntos de inflexión, los cuales expresan gráficamente la magnitud de $Lr$. Estos puntos dividen el dominio en dos zonas, la zona interior que se encuentra a distancias mayores a $Lr$ de los bordes y la zona exterior, que representa el complemento a la zona interior. La evidencia gráfica apunta a que existe una concentración de efectos no locales en la zona exterior del dominio. Esto se debe a que los elementos que se encuentren en los bordes no tendrán el mismo número de aportes de rigidez de elementos no locales como los que se obtienen en zonas interiores del dominio. \textcite{DiPaola2009} realiza una analogía con un modelo punto-resorte en una dimensión que confirma esta afirmación.

En cuanto a la elección del parámetro $l$, se debe tener en cuenta que en la literatura generalmente se enuncia que la magnitud del parámetro $l$ es pequeña (ver \cite{Eringen1972,Eringen1987}). Como se mostró en el capítulo \ref{sc:metodos} y el capítulo \ref{sc:resultados}, el hecho de que $l$ sea pequeño beneficia de manera positiva al método NL-FEM, aumentando su eficiencia computacional.

Con lo expuesto anteriormente se puede concluir que:

\begin{enumerate}
	\item Comparando los resultados obtenidos con las 4 funciones de atenuación estudiadas se puede afirmar que para valores de $l$ pequeños los resultados obtenidos son cercanos. Sin embargo, para valores de $l$ grandes los resultados de las diferentes funciones de atenuación difieren ampliamente, más aún, se agrupan dependiendo el origen de la función de atenuación. Tomando la figura \ref{fig:atenuacion_completa}, se puede afirmar que las funciones de atenuación son sensibles al parámetro $l$.
	\item Dado que la función de atenuación modificada tiene como objetivo reducir el aporte no local a la parte local, se esperaba que los resultados obtenidos por dicha función fueran lo más cercanos posibles a la solución local en el dominio interior, sin embargo, en ninguno de los resultados obtenidos se presencia este fenómeno.
	\item La elección de una función de atenuación sobre otra debe tratarse como un parámetro más del modelo. Esta elección tendrá efectos en el proceso de enmallado e integración.
	\item Al momento de elegir el parámetro $l$ se debe tener en cuenta que cuando $Lr$ no alcanza a desarrollarse, la elección de las funciones de atenuación es un factor clave en la forma de la superficie de esfuerzos.
	\item Al momento de comparar la solución de NL-FEM con la solución analítica disponible, se afirma que para todas las funciones de atenuación se incumple con la condición de que en el dominio interior las deformaciones no locales deberían ser iguales a las locales.
	\item Debido a que la función de atenuación modificada no resuelve la inconsistencia presentada en el capítulo \ref{sc:intro} se debe abordar el problema con una función de difusión distinta. Sin embargo, así como lo menciona \textcite{DiPaola2009}, esto será objetivo de investigaciones futuras.
\end{enumerate}
