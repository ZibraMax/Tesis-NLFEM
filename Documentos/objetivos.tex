\begin{comment}
\addbibresource{referencias/Referencias.bib}
\end{comment}

\section{Objetivos}
\label{sc:objetivos}

El equipo de trabajo propone que una solución a las inconsistencias presentadas en el capítulo \ref{sc:intro} es la modificación de la función de atenuación. Por tal razón, el énfasis de este trabajo es proponer una función de atenuación con el objetivo de analizar la influencia que tienen las funciones de atenuación en la solución.

De tal manera, se desarrollará una implementación de NL-FEM, sobre la que se realizarán variaciones en las funciones de atenuación. 

Concluyendo, los objetivos se pueden resumir en 3 interrogantes:
\begin{enumerate}
	\item ¿Afectan las funciones de atenuación a la solución en NL-FEM?
	\item ¿De qué manera afecta la función de atenuación a la solución?
	\item ¿Pueden funciones de atenuación distintas usarse sin alterar la solución? de ser así, ¿Se puede aprovechar este comportamiento para solventar el planteamiento del capítulo \ref{sc:objetivos}
\end{enumerate}
